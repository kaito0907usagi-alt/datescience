\documentclass{jsarticle}

\usepackage[dvipdfmx]{graphicx}
\usepackage{tabularx}
\usepackage{booktabs}
\usepackage[dvipdfmx]{xcolor}
\usepackage{listings} 

\title{データサイエンス レポート課題}
\author{25G1026 大菅 海兎}
\date{}

\lstset{
    language=Python,
    basicstyle=\ttfamily\small,
    frame={tb},
    breaklines=true,
    numbers=left,
    numberstyle=\color{gray},
    stepnumber=1,
    numbersep=5pt,
    tabsize=4,
    showspaces=false,
    xleftmargin=17pt,
    framexleftmargin=17pt,
    escapechar=@, 
}
\renewcommand{\lstlistingname}{ソースコード}

\begin{document}
\maketitle
\pagenumbering{gobble}

\subsection*{選択した分析手法}
{回帰分析}

\subsection*{例題}
2024年発売のPlayStation5用フルプライスタイトルのMetacriticによる評価と,大手中古販売会社のオンラインショップ価格をまとめた.
下記データについて回帰分析を行え.

\vspace{1zh}
\begin{table}[ht]\caption{運動前後での脈拍の測定結果}
\centering
\begin{tabular}{|c|c|c|}
\hline
       & 運動前 & 運動後 \\ \hline
被験者A & 78 & 82 \\ \hline
被験者B & 70 & 71 \\ \hline
被験者C & 73 & 74 \\ \hline
被験者D & 72 & 76 \\ \hline
被験者E & 76 & 74 \\ \hline
被験者F & 71 & 73 \\ \hline
被験者G & 77 & 82 \\ \hline
\end{tabular}
\end{table}

\subsection*{解答}
(記載例)
帰無仮説 $H_0$ は「運動前後の脈拍は等しい」である.
まず,運動前後の脈拍の差 $d$ を計算する(ここでは運動後から運動前を引く).
差 $d$ の標本平均は $\bar{d} = 2.14$,標本標準偏差は $s_d=2.41\cdots$,標本サイズは $n=7$ である.
標本標準偏差 $s_d$ の自由度は $df=n-1=6$ である.そこで,統計検定量の Student の $t$ は
$$
t = \frac{\bar{d}}{s_d / \sqrt{n}} = \frac{2.14}{2.41\cdots/ \sqrt{7}} = 2.35\cdots
$$
となる.$t$ 分布表の $df=6$ と $\alpha=0.05$ から,臨界値 $t_{0.025}(6)=2.447$ を得る.そこで,以下の不等式が成立する.
$$
2.35\cdots = |t| < t_{0.025}(6) = 2.447
$$
この結果,検定統計量 $t$ は棄却域に入らないことがわかる.
そこで,「運動前後の脈拍の間に統計的に有意な差は認められなかった」と結論する.

\begin{figure}[h]
\centering
\begin{lstlisting}[caption=プログラム例, label=src, language=Python]
    before = np.array([78, 70, 73, 72, 76, 71, 77])  # 運動前
    after  = np.array([82, 71, 74, 76, 74, 73, 82])  # 運動後
    
    # 差を計算
    d = after - before
    mean_d = np.mean(d)          # 平均差
    std_d = np.std(d, ddof=1)    # 不偏標準偏差
    n = len(d)                   # サンプルサイズ
    
    # @t検定の計算@
    t_statistic = mean_d / (std_d / np.sqrt(n))  # @t値@
    
    # @p値の計算(両側検定)@
    p_value = 2 * (1 - stats.t.cdf(abs(t_statistic), df=n-1))
    
    # 有意水準
    alpha = 0.05
    
    # 結果の表示
    print(f"@平均差: {mean\_d:.2f}@")
    print(f"@不偏標準偏差: {std\_d:.2f}@")
    print(f"@計算されたt値: {t\_statistic:.2f}@")
    print(f"@p値: {p\_value:.4f}@")
    
    if p_value < alpha:
        print("@結果: 統計的に有意です(P<0.05)(帰無仮説を棄却します).@")
    else:
        print("@結果: 統計的に有意ではありません(帰無仮説を棄却できません).@")
\end{lstlisting}
\end{figure}

\end{document}